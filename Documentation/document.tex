\documentclass[pagesize=pdftex,paper=a4,fontsize=12pt]{scrartcl}


\usepackage[ngerman]{babel} % Neue Rechtschreibung, Inhaltsverzeichnis, ...
\usepackage[utf8]{inputenc} % Umlaute
\usepackage[T1]{fontenc} % vorgefertigte zusammengesetzte Zeichen
\usepackage{lmodern}
\usepackage{amsmath} % Mathematikumgebung
\usepackage{amssymb} % Mathsymbole
\usepackage{array} % Tabellen
\usepackage{graphicx} % Bilder einfügen
\usepackage[ngerman]{babel}
\usepackage[babel, german=quotes]{csquotes}
\usepackage{hyperref}
\usepackage{pgfplots}
\usepackage{siunitx}
\usepackage{svg}

\usepackage{tikz} % Grafiken
\usetikzlibrary{arrows,decorations.pathmorphing}

\title{Rainbows}
\author{Johannes Wilde}

\newcommand{\R}{\mathbb{R}}

\setlength{\oddsidemargin}{-1cm}
\setlength{\textwidth}{18cm}
\setlength{\topmargin}{-1.5cm}
\setlength{\textheight}{24cm}

\begin{document}

\maketitle

\section{Geometrical Assumptions}



\begin{figure}[h!]
	\centering
	\includesvg[width=15cm]{PrincipleSketch.svg}
	\caption{Principle scetch of the geometrical properties of a circular raindrop.}
	\label{fig:PrincipleScetch}
\end{figure}


\[ h = \sin(\alpha) \]

\[ n_0 \sin(\alpha) = n_1 \sin(\beta) \]

\[ \varepsilon = \alpha - \beta \]

\[ \zeta = [\SI{180}{\degree} - 2 \beta] / 2 = \SI{90}{\degree} - \beta \]

\[ \delta = \SI{180}{\degree} - [[\SI{180}{\degree} - 2 \beta] + \varepsilon] = 2 \beta - [\alpha - \beta] = 3 \beta - \alpha \]

\[ \gamma = \SI{180}{\degree} - 2 [\zeta + \varepsilon] = \SI{180}{\degree} - 2 [\SI{90}{\degree} - \beta + \alpha - \beta] = 2 [2 \beta - \alpha] \]

\begin{align}
	\gamma (\alpha) & = 2 \left[2 \arcsin\left(\frac{n_0}{n_1} \sin(\alpha)\right) - \alpha\right] \nonumber \\[.5em]
	\gamma (h) & = 2 \left[2 \arcsin\left(\frac{n_0}{n_1} h\right) - \arcsin(h)\right]
\end{align}

\begin{figure}[h!]
	\centering
	\begin{tikzpicture}
	\begin{axis}[
		scale=1.5,
		domain=-1:1,
		samples=500,
		xlabel={$h\,[\text{normalized}]$},
		ylabel={$\gamma\,[\si{\degree}]$},
		legend pos=north west,
		legend columns=2,
		cycle list name=color list
	]
		\foreach \refractiveIndexRelation in {1.00, 1.05, 1.10, 1.15, 1.20, 1.30, 1.40, 1.60, 1.80, 2.00, 2.40, 2.80}
		{
			\addplot +[mark=none] (x, {2 * (2 * asin(x/\refractiveIndexRelation) - asin(x))});
			\expandafter\addlegendentry\expandafter{\expandafter$\refractiveIndexRelation\expandafter$};
		}
	\end{axis}
	\end{tikzpicture}
	\caption{Excidence angle $\gamma$ over incidence height $h$.}
	\label{fig:GammaOverH}
\end{figure}


\section{Extrema}

Extrema [$\frac{d}{dx} \arcsin(x) = \frac{1}{\sqrt{1-x^2}}$]

\[ \frac{d}{dh} \gamma (h) = 2 \left[2 \frac{n_0}{n_1} \frac{1}{\sqrt{1-[\frac{n_0}{n_1} h]^2}} - \frac{1}{\sqrt{1-h^2}}\right] \]

\newcommand{\hExtremum}{\ensuremath{h_{\text{extr.}}}}
\newcommand{\gammaExtremum}{\ensuremath{\gamma_{\text{extr.}}}}
\begin{align}
	0 & \stackrel{!}{=} \frac{d}{dh} \gamma (\hExtremum)\\
 & = 2 \left[2 \frac{n_0}{n_1} \frac{1}{\sqrt{1-[\frac{n_0}{n_1} h]^2}} - \frac{1}{\sqrt{1-h^2}}\right] \nonumber \\[.5em]
	\frac{1}{\sqrt{1-h^2}} &= 2 \frac{n_0}{n_1} \frac{1}{\sqrt{1-[\frac{n_0}{n_1} \hExtremum]^2}} \nonumber \\[.5em]
	1-\left[\frac{n_0}{n_1} \hExtremum\right]^2 &= \left[2 \frac{n_0}{n_1}\right]^2 \left[1-\hExtremum^2\right] \nonumber \\[.5em]
	\left[\frac{n_1}{2 n_0}\right]^2 -\left[\frac{1}{2} \hExtremum\right]^2 &=  1-\hExtremum^2 \nonumber \\[.5em]
	\frac{3}{4} \hExtremum^2 &=  1 - \left[\frac{n_1}{2 n_0}\right]^2 \nonumber \\[.5em]
	\hExtremum^{\pm} &= \pm \sqrt{\frac{1}{3} \left[4 - \left[\frac{n_1}{n_0}\right]^2\right]}
\end{align}

\begin{equation}
	 1 < \frac{n_1}{n_0} \leq 2
\end{equation}

\begin{equation}
	\gammaExtremum^{\pm} = \gamma (\hExtremum^{\pm}) = \pm 2 \left[2 \arcsin\left(\frac{n_0}{n_1} \sqrt{\frac{1}{3} \left[4 - \left[\frac{n_1}{n_0}\right]^2\right]}\right) - \arcsin\left(\sqrt{\frac{1}{3} \left[4 - \left[\frac{n_1}{n_0}\right]^2\right]}\right)\right]
\end{equation}


\begin{figure}[h!]
	\centering
	\begin{tikzpicture}
	\begin{axis}[
	scale=1.5,
	domain=1:2,
	samples=500,
	xlabel={$\frac{n_1}{n_0}$},
	ylabel={$\gammaExtremum^{+}\,[\si{\degree}]$},
	legend pos=north west,
	legend columns=1,
	cycle list name=color list
	]
		\addplot[mark=none] (x, { 2 * (2 * asin(sqrt((4-x^2)/3)/x) - asin(sqrt((4-x^2)/3))) });
		\node[label={0:{water @ \SI{550}{\nano\meter}}},circle,fill,inner sep=2pt] at (axis cs:1.334683329053798,41.83394805963984) {};
	\end{axis}
	\end{tikzpicture}
	\caption{Excidence angle $\gamma{+}$ over incidence height $h$.}
	\label{fig:GammaMaxOverRefractiveIndexRelation}
\end{figure}

For water at \SI{20}{\degreeCelsius} and light of \SI{550}{\nano\meter} [$n_0 = 1$, $n_1 = 1.3347$] this yields an angle of $\gammaExtremum^{+} = \SI{41.83}{\degree}$


\section{Power Density Superelevation}

\end{document}